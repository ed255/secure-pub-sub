\section{Related Work}
\label{sec:related}

Nikolaenko, et al., \cite{nikolaenko2013privacy} proposed a scalable
privacy-preserving system for ridge-regression combining additive homomorphic
encryption and Yao's garbled circuits.  In their setting, a single evaluator is
interested in learning ridge regression over data of a large number of data
owners without learning the individual data of data owners. Our setting and
approach is different: (\emph{a.}) we don't want to reveal output to the
evaluator, (\emph{b.}) we have multiple subscribers who want the output of
computation, (\emph{c.}) our data owners, publishers, are oblivious of
subscribers and subscribers are oblivious of publishers, (\emph{d.}) we support
arbitrary polynomial-time computations as opposed to just ridge regression, and
(\emph{e.}) we don't necessarily need a third party \garbler, as show in
Figure~\ref{fig:pps-local} if a set subscribers controlled by a single entity
can do a modest amount of computation. Nikolaenko, et
al.,\cite{nikolaenko2013privacy} proposed a similar system but for
privacy-preserving matrix factoring.

Naveed, et al., \cite{naveed2014controlled} proposed a new primitive called
controlled functional encryption inspired by functional encryption. Instead of
using the same key for a function for computing the function over multiple
ciphertexts, controlled functional encryption, require a fresh function key for
every ciphertext. They construct an efficient controlled functional encryption
scheme for arbitrary polynomial time computations based on Yao's garbled
circuits and CCA2 secure public key encryption. Their setting involves a data
owner who wants a client to perform specific computations on it's data with the
help of an online key authority. Their setting is similar to our setting of
Figure~\ref{fig:pps-local}; however, in our setting publishers and subscribers
are oblivious to each other.

Fully homomorphic encryption allows arbitrary computation on encrypted data.
Gentry proposed the first fully homomorphic encryption
scheme \cite{brakerski2011fully,gentry2009fully} followed by several more
efficient schemes \cite{brakerski2014leveled}. Dijk, et
al., \cite{van2010impossibility} showed that privacy-preserving outsourced
computation on data from multiple parties and supplying output to multiple
parties, that is our setting, requires, in addition to fully homomorphic
encryption, access-controlled ciphertexts and re-encryption. They reduce a
scheme that computes on data from two parties and supply the output to two
parties to black box obfuscation, which is impossible in
general \cite{barak2001possibility}.
