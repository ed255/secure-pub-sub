\section{Introduction}
%Describe MQTT protocol

\section{Related Work}

\section{Protocol}

* We minimize the communication with \garbler, using clever tricks, e.g.,
we don't need to send labels to \garbler for circuit garbling instead they
generate it independently using shared seed.

* publishers and subscribers only talk to the broker. The communication
between publishers and subscriber is done through \broker using end-to-end
encryption. This ensures that 

* Does the subscriber seed needs to be computation specific?

\noindent\textbf{Initialization.}
\begin{itemize}[leftmargin=*]
 
  \item Each new publisher: 
		\begin{itemize}
			\item sends its public key $pk$ to \broker and \garbler and obtains
		  public keys $pk_B$ and $pk_C$ from \broker and \garbler.
			\item sends a policy specifying allowed computations on its data.
			\item generates a truly random seed $s_P$ and send it to \garbler
		  through authenticated and encrypted channel using \garbler public
		  key $pk_C$. We use ratcheting for forward security if seed is
		  compromised.
		\end{itemize} 

	\item Each new subscriber:
		\begin{itemize}
			 \item sends its public key $pk'$ to \broker and \garbler.
		\end{itemize}

\end{itemize}

\noindent\textbf{Subscription.}
\begin{itemize}[leftmargin=*]

	\item To subscribe computation $c$, subscriber sends a subscription request
	  containing $c$ to \broker and requests an output masking seed from
	  \garbler.  
	\item \garbler sends a truly random seed $s'_c$ for computation $c$;
	  generating a new seed if this is the first subscription for computation
	  $c$. 
\end{itemize}

\noindent\textbf{Publication.}
\begin{itemize}[leftmargin=*]
		
	\item To publish $k$th value, publisher generates two pseudorandom wire
	  labels, $w_0$ and $w_1$, using seed $s_i$, for each bit of the value.
	  $w_0$ is $a$th and $w_1$ is $(a+1)$th numbers in pseudorandom sequence
	  generated using seed $s_i$; $2kL \leq a < 2(k+1)L$, $L$ being the
	  bit-length of a value.

\end{itemize}

\noindent\textbf{Computation.}
\begin{itemize}[leftmargin=*]

  \item After \broker has wire labels for all publishers' inputs required for
	computation $c$, it requests \garbler to garble circuit for $c$.  
  
  \item \garbler independently generates input wire labels using seed $s_P$ and
	an output mask $m$ using seed $s'_c$ for each bit of output.

  \item \garbler generates garbled circuit for $M \circ C(.)$, composition of
	masking function $M$ and computation $C$, and sends it to \broker.

  \item \broker evaluates the garbled circuit using wire labels sent by
	publishers, obtains masked output $o \xor m$, and send $o \xor m$ to all
	subscribers of computation $c$.
  
  \item Subscribers generate the mask $m$ using the seed $s'_c$ and unmask the
	output $o$.

\end{itemize}


* What if one publisher doesn't send wire labels. After timeout the broker can inform the \garbler and it will use zero for such values in the circuit.

\section{System}

\section{Evaluation}

\section{Conclusions}

%\appendix
%Note that in the new ACM style, the Appendices come before the References.

\begin{acks}
% TODO: For the submission, don't include acknowledgments since they would most
	% likely deanonymize you.
\end{acks}
