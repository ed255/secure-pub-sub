\section{Security Definition}
\label{sec:definition}

\improvement{Explain how pub/sub works}
Before defining security, we present threat model that our definition needs to
capture.

\improvement{Explain later that publishers cannot control which subscribers can
subscribed and similary susbcriber cannot determine which publishers will be
used. Publisher either allows data for all subscribvers or noone at all; this
is how pub/sub model works. However, \broker can control which entieies can
subscribe.}

\vspace{-10pt}
\subsection{Threat Model} We assume that the adversary can \emph{maliciously}
compromise a subset of subscribers and a subset of publishers and
\emph{passively} compromise either \broker or \garbler, but not both.
Malicious subscribers and publishers can collude either with \garbler or
\broker, but not both.

\vspace{-10pt}
\subsection{Definition}

We define security for a secure publish-process-subscribe protocol in
REAL/IDEAL paradigm. Our definition captures both confidentilaity and
correctness against adversary discussed above in the threat model.\\[6pt]
\noindent\textbf{Ideal World.}
Our ideal functionality \F interacts with publishers, subscribers, \broker, and
\garbler as follows:

\begin{itemize}[leftmargin=*,itemsep=4pt,topsep=4pt]
		\item Each publisher sends a \policy to \F and each subscriber sends a
			subscription message to \F containing subscribed computation $C$. Let $S_C$
			be the set of $C$'s subscribers. 
			
%		Let $p_0, \ldots, p_{m-1}$ be the publishers whose data is required to
			%		compute $C$. 
		
		\item The honest publishers upload data to \F. The malicious publishers
			controlled by \Adv may abort (by not sending anything), sends their
			actual input, or send any arbitrary value that may depend upon their
			input, other malicious publishers' input, and auxiliary data. If a
			publisher value is not received before a time out period, \F uses a
			nullifying value for its input, e.g., $0$ for addition and $1$ for
			multiplication. 
			%We call $i$th publisher's input $x_i$; $0 \leq i < m$, $m$ being the
			%total number of publishers. 
		
		\item \garbler uploads to \F a one-time use pseudorandom mask $r$.

		\item \F determines a subset $P' \subset P$ of publishers whose data can be
			used to compute $C$. \F sends to \broker $P'$, policies of all publishers
			in $P'$, and $C$.

		\item \broker sends \F a subset $P_C \subset P'$ of publishers whose
			policies allow $C$.
			
		\item If data $\vec{x}_C$ of all publishers in $P_C$ is enough to compute
			$C$, \F sends $C(\vec{x}_C) \xor r$ and $r$ to all subscribers in $S_C$,
			otherwise \F sends empty message $\bot$ to all subscribers in $S_C$.

\end{itemize}
\vspace{6pt}
\noindent\textbf{Real World.} In real world, \F is replaced by our
protocol described in Figure~\ref{fig:baseprotocol}.\\

\begin{mdframed}[style=mydefframe]

\begin{define}
	\label{def:security}

	\textit{	A publish-process-subscribe protocol is simulation secure if for
	every adversary \Adv in the real world that maliciously corrupts a subset of
	publishers and a subset of subscribers, and only passively corrupts \broker
	and \garbler, with arbitrary collusion between malicious publishers,
	malicious subscribers, and hones-but-curious \broker and \emph{no} collusion
	between \broker and \garbler, there exists a simulator \Sim in the ideal
	world that also corrupts the same set of parties and produces an output
	identically distributed to \Adv's output in the real world.  
	}

\end{define}

\end{mdframed}
