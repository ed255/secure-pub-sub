%\improvement{Inconsistent wire labels}
%
%The intereseting case is when publishers are corrupt and \broker and \garbler
%are hones-but-curious.
%
%\Adv sends \Sim inputs of all malicious publishers. \Sim uses zero inputs for
%honest publishers. \Sim generates wire labels for all inputs 
%
%\Sim sends all inputs and output mask $r$ to \F and learns $\F(M \circ C,
%\vec{x})$, whre $M$ is an XOR masking function. \Sim sends $\F(M \circ C,
%\vec{x})$ to $\Sim_{GC}$ and obtains a fake garbled $GC_{fake}$. \Sim
%generates a single random wire label for each wire of the circuit. \Sim sends
%$GC_{fake}$ and wire labels to \Adv.  $\Sim_{GC}$ is computationally
%indistinguishable from the garbled circuit generated in the real execution, as
%garbled circuit distribution doesn't depend upon on the input values. The
%pseudorandom wire labels are also computational indistinguishable from actual
%wire labels.
%
%This proof implies both privacy and correctness as distribution in both the
%real and ideal world have a negligible difference. 
%
%
%\Sim generates random wire labels for all wires of the circuit $M \circ C$. 
%
%\Sim receives from \F a handle $h_C$ and the number of publishers $|P_C|$
%whose policy allow $C$. \Sim

We describe a simulator \Sim that simulates the view of the \Adv in the real
world execution of our basic protocol of Figure~\ref{fig:basicprotocol}. Our
security definition~\ref{def:security} and simulator \Sim ensures both
confidentiality and correctness.

The interesting case is when a subset of publishers are malicious and are
colluding with honest-but-curious \broker. \garbler is non-colluding
honest-but-curious. A malicious subscriber only receives output; it can only
reveal its own output, which is also possible in the ideal world execution.\\

\noindent\textbf{Simulator.} \Sim receives from \F the number of publishers
$|P_C|$ whose policy allow computing $C$ on their data. \Sim creates $2l|P_C|$
random wire labels $(r_0^0, r_0^1), \ldots, (r_{2l|P_C|-1}^0
,r_{2l|P_C|-1}^1)$; $l$ being the bit-length of a publisher's input. We use
garbled circuit simulator $\Sim_{GC}$ as a blackbox; $\Sim_{GC}$ is the
simulator of the projective prv.sim secure garbling scheme with circuit $M
\circ C$ being the side information as described in~\cite{}. 

\Sim receives from \F $\F(M \circ C, \vec{x}_C)$, where $M$ is an XOR masking
function. \Sim sends $\F(M \circ C, \vec{x}_C)$ to $\Sim_{GC}$ and obtains a
fake garbled $GC_{fake}$. \Sim generates a random string $o_r$ of the same
length as output. \Sim sends $(GC_{fake}, r_0^0, \ldots, r_{2l|P_C|-1}^0, o_r)$
to \Adv. As garbled circuits distribution is independent of the input wire
labels, $GC_{fake}$ is computationally indistinguishable from the $GC$ in the
real execution. The random output $o_r$ in ideal execution is indistinguishable
from $o+r$ in the real execution.

In the ideal world, \Sim creates a fake garbled circuit $GC_{fake}$ that
doesn't use wire labels $(r_0^0, r_0^1), \ldots, (r_{2l|P_C|-1}^0
,r_{2l|P_C|-1}^1)$ for garbling. Otherwise, \Adv could use $r_0^0, \ldots,
r_{2l|P_C|-1}^0$ labels to evaluate the circuit on $0^{l|P_C|}$, which would
allow adversary to distinguish between real and ideal executions.

A malicious publisher can choose arbitrary wire labels in the real execution;
however, as long as the labels used in garbling are consistent with the labels
used for evluation, the honest subscriber output will be indistinguishable in
real and ideal executions. Our protocol ensures consistent wire labels. 

%\Sim generates a single random wire label for each
%wire of the circuit. \Sim sends $GC_{fake}$ and wire labels to \Adv.
%$\Sim_{GC}$ is computationally indistinguishable from the garbled circuit
%generated in the real execution, as garbled circuit distribution doesn't depend
%upon on the input values. The pseudorandom wire labels are also computational
%indistinguishable from actual wire labels.
%
%
%We ensure such consistency
%using private set intersection cardinarlity between  
%
%
%
%
%In the real execution, a
%malicious publisher can send inconsistent wire labels to \broker and \garbler;
%which will result in \broker being unable to evalute the garbled circuit. \broker
%and \garbler use private set intersection cardinality to ensure that \broker and \garbler labels are consistent. 
%
%A malicious publisher may send
%non-random wire labels or inconsistent wire labels. To make the wire labels
%consistent, in the real execution, \broker and \garbler use private set
%intersection cardinality to figure out if \broker has a consistent label. As
%long as this property holds, the value of the labels doesn't affect the output
%of the honest parties. 
%
%
%A malicious publisher may not
%send a random wire label in the real execution.
