\subsection{Security Definition}

\improvement{Explain later that publishers cannot control which subscribers can
subscribed and similary susbcriber cannot determine which publishers will be
used. Publisher either allows data for all subscribvers or noone at all; this
is how pub/sub model works. However, \broker can control which entieies can
subscribe.}

\paragraph{Ideal World.}
Our ideal functionality \F interacts with publishers, subscribers, \broker, and
\garbler as follows:

\begin{itemize}[leftmargin=*]
		\item Each publisher sends a \policy to \F and each subscriber sends a
			subscription message to \F containing subscribed computation $C$. Let $S_C$
			be the set of $C$'s subscribers. 
			
%		Let $p_0, \ldots, p_{m-1}$ be the publishers whose data is required to
			%		compute $C$. 
		
		\item The honest publishers upload data to \F. The malicious publishers
			controlled by \Adv may abort (by not sending anything), sends their
			actual input, or send any arbitrary value that may depend upon their
			input, other malicious publishers' input, and auxiliary data. If a
			publisher value is not received before a time out period, \F uses a
			nullifying value for its input, e.g., $0$ for addition and $1$ for
			multiplication. 
			%We call $i$th publisher's input $x_i$; $0 \leq i < m$, $m$ being the
			%total number of publishers. 
		
		\item \garbler uploads to \F a one-time use pseudorandom mask $r$.

		\item \F determines a subset $P' \subset P$ of publishers whose data can be
			used to compute $C$. \F sends to \broker $P'$, policies of all publishers
			in $P'$, and $C$.

		\item \broker sends \F a subset $P_C \subset P'$ of publishers whose
			policies allow $C$.
			
		\item If data $x_C$ of all publishers in $P_C$ is enough to compute $C$, \F
			sends $C(\vec{x_C}) \xor r$ and $r$ to all subscribers in $S_C$,
			otherwise \F sends empty message $\bot$ to all subscribers in $S_C$. 

\end{itemize}

\paragraph{Real World.} In real world, \F is replaced by our protocol~\ref{}.

\noindent\textbf{Simulator.}
%\improvement{Inconsistent wire labels}
%
%The intereseting case is when publishers are corrupt and \broker and \garbler are hones-but-curious.
%
%\Adv sends \Sim inputs of all malicious publishers. \Sim uses zero inputs for
%honest publishers. \Sim generates wire labels for all inputs 
%
%\Sim sends all inputs and output mask $r$ to \F and learns
%$\F(M \circ C, \vec{x})$, whre $M$ is an XOR masking function. \Sim sends $\F(M
%\circ C, \vec{x})$ to $\Sim_{GC}$ and obtains a fake garbled $GC_{fake}$. \Sim
%generates a single random wire label for each wire of the circuit. \Sim sends
%$GC_{fake}$ and wire labels to \Adv.  $\Sim_{GC}$ is computationally
%indistinguishable from the garbled circuit generated in the real execution, as
%garbled circuit distribution doesn't depend upon on the input values. The
%pseudorandom wire labels are also computational indistinguishable from actual
%wire labels.
%
%This proof implies both privacy and correctness as distribution in both the
%real and ideal world have a negligible difference. 
%
%
%\Sim generates random wire labels for all wires of the circuit $M \circ C$. 
%
%\Sim receives from \F a handle $h_C$ and the number of publishers $|P_C|$ whose
%policy allow $C$. \Sim 
The interesting case is when up to all-but-one publishers are malicious and are
colluding with honest-but-curious \broker. A malicous \broker affects
correctness but not privacy. A malicious subscriber only receives output; it
can only reveal its own output, which is possible in ideal execution as well.

\Sim receives from \F the number of publishers $|P_C|$ whose policy allow
computing $C$ on their data. \Sim creates $2l|P_C|$ random wire labels $(r_0^0,
r_0^1), \ldots, (r_{2l|P_C|-1}^0 ,r_{2l|P_C|-1}^1)$; $l$ being the bit-length
of a publisher's input. We use garbled circuit simulator $\Sim_{GC}$ as a
blackbox; $\Sim_{GC}$ is the simulator of the projective prv.sim secure
garbling scheme with circuit $M \circ C$ being the side information as
described in~\cite{}. 

\Sim receives from \F $\F(M \circ C, \vec{x_C})$, where $M$ is an XOR masking
function. \Sim sends $\F(M \circ C, \vec{x_C})$ to $\Sim_{GC}$ and obtains a
fake garbled $GC_{fake}$. \Sim generates a random string $o_r$ of the same
length as output. \Sim sends $(GC_{fake}, r_0^0, \ldots, r_{2l|P_C|-1}^0, o_r)$
to \Adv. As garbled circuits distribution is indepedent of the input wire
labels, $GC_{fake}$ is computationally indistinguishable from the $GC$ in the
real execution. The random output $o_r$ in ideal execution is indistinguishable
from $o+r$ in the real execution.

In the ideal world, \Sim creates a fake garbled circuit $GC_{fake}$ that
doesn't use wire labels $(r_0^0, r_0^1), \ldots, (r_{2l|P_C|-1}^0
,r_{2l|P_C|-1}^1)$ for garbling. Otherwise, \Adv could use $r_0^0, \ldots,
r_{2l|P_C|-1}^0$ labels to evaluate the circuit on $0^{l|P_C|}$, which would
allow adversary to distinguish between real and ideal executions.

A malicious publisher can choose arbitrary wire labels in the real execution;
however, as long as the labels used in garbling are consistent with the labels
used for evluation, the honest subscriber output will be indistinguishable in
real and ideal executions. Our protocol ensures consistent wire labels. 

Our threat model allows for publishers and subscribers to collude aribitrailly.
\broker and \garbler are semihonest.


%\Sim generates a single random wire label for each
%wire of the circuit. \Sim sends $GC_{fake}$ and wire labels to \Adv.
%$\Sim_{GC}$ is computationally indistinguishable from the garbled circuit
%generated in the real execution, as garbled circuit distribution doesn't depend
%upon on the input values. The pseudorandom wire labels are also computational
%indistinguishable from actual wire labels.
%
%
%We ensure such consistency
%using private set intersection cardinarlity between  
%
%
%
%
%In the real execution, a
%malicious publisher can send inconsistent wire labels to \broker and \garbler;
%which will result in \broker being unable to evalute the garbled circuit. \broker
%and \garbler use private set intersection cardinality to ensure that \broker and \garbler labels are consistent. 
%
%A malicious publisher may send
%non-random wire labels or inconsistent wire labels. To make the wire labels
%consistent, in the real execution, \broker and \garbler use private set
%intersection cardinality to figure out if \broker has a consistent label. As
%long as this property holds, the value of the labels doesn't affect the output
%of the honest parties. 
%
%
%A malicious publisher may not
%send a random wire label in the real execution.
