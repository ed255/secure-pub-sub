\section{Introduction}
\label{sec:intro}

As the Internet of Things (IoT) is emerging, rich new sets of applications are
being envisioned that will require large-scale and complex deployments. These
are motivating the rise of new network and middleware protocols to address key
challenges including scale, inter-operability, security, and
robustness~\cite{Stankovic14, al2015internet, sicari2015security}. 

For example, smart cities concepts are being proposed that incorporate many
IoT-based deployments and applications. These include smart transportation
applications that make use of real time sensor feeds from road traffic sensors
(both infrastructure based and participatory mobile
sensing)~\cite{leduc2008road, mohan2008nericell}, smart parking
meters~\cite{ji2014cloud}, and bike sharing systems that utilize sensor feeds
on location of bikes~\cite{midgley2009role}; applications aimed at monitoring
and improving the environment using static and mobile sensors to measure air
and water quality~\cite{dutta2009common, le2007design}; automated water and
power meter reading applications~\cite{khalifa2011survey}; smart sanitation
applications that use streams of sensor data from smart
trash~\cite{glouche2013smart} cans indicating their utilization levels for
timely removal of urban trash, etc. 

The first generation of IoT systems involving large numbers of densely-deployed
real-time streaming sensors such as traffic road sensor
deployments~\cite{pems17}, and air quality measurement~\cite{aqmd17} have
largely focused on monolithic, vertically integrated applications with
essentially a single collection point (which serves as an independent
application-specific gateway for other applications to pull data from, in some
cases through the provision of suitable API's, or simply through web-scraping). 

However, there is a growing recognition that it is desirable to utilize
application-agnostic middleware that allows the number of consumption points to
scale. Thus, there has been a concerted effort to build and deploy IoT
applications using real-time multi-point to multi-point middleware (that sits
above the transport layer), and the most commonly used pattern for this kind of
communication is the publish-subscribe paradigm~\cite{pubsub}.
Publish-Subscribe middleware allow multiple data consumers to connect to and
receive streams of real time data from large numbers of sensors. Commonly used
examples of publish-subscribe middleware protocols are Advanced Message Queuing
Protocol (AMQP)~\cite{amqp} and Message Queue Telemetry Transport (MQTT)~\cite{mqtt}, and
also on the rise are commercial publish-subscribe platform as a service (PaaS)
providers such as PubNub~\cite{pubnub} which provide their own proprietary
protocol and API.  

The basic idea behind publish-subscribe systems such as MQTT is that there is a
broker (typically centralized and implemented on a cloud server, such as the
mosquitto broker for MQTT~\cite{light2013mosquitto}, though there are also some
distributed server implementations). Sensors publish messages to specified
"topics" that are sent to the broker. Data consumers send to the broker a
subscribe request to the specified topics by name. Following this, all messages
published to any topic are forwarded by the broker to each subscriber of that
topic. While we focus in this paper on sensing-oriented applications, note that
pub-sub middleware could also be used to connect applications that generate
control actions with actuators (in this case the actuator devices would be the
subscribers and controlling applications would be the publishers). As one
concrete example of the benefits of a pub-sub system, consider its possible use
in a bike-sharing system; signals from bikes that are available could be
published to a topic defined by spatial region (e.g. road stretch), and
potential riders' devices can subscribe to topics corresponding to the
locations near them, allowing them to find the location where they can pick up
an available bike.

The broker in a traditional pub-sub system such as MQTT plays primarily an
application layer forwarding role. It may also be enhanced to do some basic
authentication of subscribers and publishers to ensure that only authorized
users have access to send/receive relevant raw data streams from the broker. 

More than a decade of research, particularly into  wireless sensor
networks~\cite{yao2002cougar, krishnamachari2002impact}, has shown the value of
incorporating in-network aggregation and processing into sensing applications.
In-network processing allows for greater throughput and energy efficiency
(through compression), more meaningful information to derived from raw sensors
(through collection of key statistics, multi-modal sensor fusion), anomaly
detection, privacy (by allowing aggregation and anonymization of raw data).
Prior work has largely addressed incorporating in-network aggregation in the
context of data-centric pub-sub systems that operate at the network-layer, but
there is a growing recognition that pub-sub middleware at the application layer
can also benefit greatly from incorporating processing at the intermediate
broker. In this new emerging  publish-process-subscribe paradigm, the broker
not only serves as a relay, but also enables computations over the raw streams
so that the end subscriber receives real-time computed versions over one or
more published streams~\cite{kwame17}. This architecture is particularly
appealing as it reduces bandwidth consumption and doesn't place computation on
end points which may be resource constrained due to need for low-power and
low-cost. Evidence of the growing appeal of this architecture is that the
commercial PaaS pub-sub provider PubNub recently introduced processing
capability into their system in the form of real-time compute functions they
refer to as "BLOCKS"~\cite{blocks}.

However, as brokers are typically hosted at third party servers, adding
computational processing to pub-sub middleware  introduces concerns about
confidentiality. An application that wishes to make use of a third-party MQTT
server for traditional pub-sub messaging could always encrypt the data end to
end to provide confidentiality guarantees. But with computational functionality
being moved to the server, these guarantees no longer hold. This is the focus
of our work.\\[6pt]
\noindent\textbf{Contributions.}

\begin{itemize}[leftmargin=*,itemsep=4pt,topsep=4pt]

	\item We propose the first provably-secure publish-process-subscribe
		protocol. We also present a security definition for a secure
		publish-process-subscribe protocol in REAL/IDEAL paradigm and a
		simulation based security proof.

	\item Using our protocol as a foundation, we develop a full-fledged
		publish-process-subscribe system with novel system components and report a
		comprehensive evaluation with real applications.

\end{itemize}
%now describe what this paper is about, contributions, etc. 

\noindent\textbf{Organization.} We describe related work in
Section~\ref{sec:related}, propose a security definition for a
publish-process-subscribe protocol in Section~\ref{sec:definition}, detail our
protocol along with extensions and security proof in
Section~\ref{sec:protocol}, describe our system in Section~\ref{sec:system},
report evaluation using real applications in Section~\ref{sec:evaluation}, and
conclude in Section~\ref{sec:conclusion}.





