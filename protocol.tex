\section{Protocol}
\label{sec:protocol}

We describe our protocol. We describe system in Section~\ref{def:system}, and
we solve many issues that come into building a system.

\improvement{Malicious \broker doesn't affect privacy, online correctness}

\subsection{Basic Protocol}
\input{basic_protocol}

\subsection{Security Proof of Basic Protocol}
\noindent\textbf{Simulator.} \Sim receives from \F the number of publishers
$|P_C|$ whose policy allow computing $C$ on their data. \Sim creates $2l|P_C|$
random wire labels $(r_0^0, r_0^1), \ldots, (r_{2l|P_C|-1}^0
,r_{2l|P_C|-1}^1)$; $l$ being the bit-length of a publisher's input. We use
garbled circuit simulator $\Sim_{GC}$ as a blackbox; $\Sim_{GC}$ is the
simulator of the projective prv.sim secure garbling scheme with circuit $M
\circ C$ being the side information as described in~\cite{}. 

\Sim receives from \F $\F(M \circ C, \vec{x}_C)$, where $M$ is an XOR masking
function. \Sim sends $\F(M \circ C, \vec{x}_C)$ to $\Sim_{GC}$ and obtains a
fake garbled $GC_{fake}$. \Sim generates a random string $o_r$ of the same
length as output. \Sim sends $(GC_{fake}, r_0^0, \ldots, r_{2l|P_C|-1}^0, o_r)$
to \Adv. As garbled circuits distribution is independent of the input wire
labels, $GC_{fake}$ is computationally indistinguishable from the $GC$ in the
real execution. The random output $o_r$ in ideal execution is indistinguishable
from $o+r$ in the real execution.

In the ideal world, \Sim creates a fake garbled circuit $GC_{fake}$ that
doesn't use wire labels $(r_0^0, r_0^1), \ldots, (r_{2l|P_C|-1}^0
,r_{2l|P_C|-1}^1)$ for garbling. Otherwise, \Adv could use $r_0^0, \ldots,
r_{2l|P_C|-1}^0$ labels to evaluate the circuit on $0^{l|P_C|}$, which would
allow adversary to distinguish between real and ideal executions.

A malicious publisher can choose arbitrary wire labels in the real execution;
however, as long as the labels used in garbling are consistent with the labels
used for evluation, the honest subscriber output will be indistinguishable in
real and ideal executions. Our protocol ensures consistent wire labels.


\subsection{Reduced Communication Extension}
\begin{figure}
\begin{mdframed}[style=myframe]

\initialize
\begin{itemize}[leftmargin=*,itemsep=2pt,topsep=2pt]
 
	\item Each new publisher generates and sends to \garbler a truly random seed
		$s$. This seed will be used to create wire labels without interaction.

\end{itemize}

\subscribe
\begin{itemize}[leftmargin=*,itemsep=2pt,topsep=2pt]

	\item In addition to registering subscription with \broker, subscribers for a
		computation $C$ also register with \garbler. \garbler sends a truly
		random seed $s'$ for computation $C$ and send it to every subscriber who
		subscribes for $C$; generating a new seed for the first subscription for
		computation $C$.
		
\end{itemize}

\publish
\begin{itemize}[leftmargin=*,itemsep=2pt,topsep=2pt]
		
	\item To publish $k$th value, publisher generates two pseudorandom wire
		labels, $w_0$ and $w_1$, using seed $s$ in a pseudorandom number generator
		(PRNG), for each bit of the value.  $w_0$ is $i$th and $w_1$ is $(i+1)$th
		numbers in pseudorandom sequence generated using seed $s$; $2kL \leq i <
		2(k+1)L$, $L$ being the bit-length of a value.

	\item For each input bit $b$, publisher sends only wire label $w_b$ to
		\broker.

\end{itemize}

\process
\begin{itemize}[leftmargin=*,itemsep=2pt,topsep=2pt]

	\item \garbler independently generates input wire labels using seed $s$ from
		each publisher contributing input and an output mask $r$ using seed $s'$
		for the output.

\end{itemize}

\vspace{2pt}

\textbf{Forward Secure Seeds}
\vspace{2pt}

The following procedure ensures that seeds $s$ and $s'$ used above are forward
	secure, i.e., compromise of seed does not affect the confidentiality of past
	data. We adapt Signal, a popular secure messaging protocol, key ratcheting
	protocol for forward security.

\begin{itemize}[leftmargin=*,itemsep=2pt,topsep=2pt]

	\item Generate a truly random key $K_0$.

	\item Generate, using pseudorandom function (PRF) with key $K_0$, a
		pseudorandom seed $s_0$ and a pseudorandom key for the ratchet round $1$.
		Seed $s_0$ is used to generate pseudorandom strings during ratchet round
		$0$.

	\item At round $i$, using PRF with key $K_i$, generate a pseudorandom seed
		$s_i$ and key for ratchet round $i+1$. Seed $s_i$ is used to generate
		pseudorandom strings during ratchet round $i$.

\end{itemize}

\end{mdframed}
\caption{Reduced Communication Extension with Forward Security}
\label{fig:extended_protocol}
\end{figure}


\subsection{Efficient Wire Labels Consistency Check} In our basic
protocol, described in Figure~\ref{fig:basicprotocol}, we execute a new
instance of private set intersection \emph{cardinality} protocol for every
input bit of the circuit, which is expensive.  We present a more efficient wire
label consistency protocol:

For every input bit $b$ of circuit $C$, \garbler has both wire labels $w_0$ and
$w_1$, we call the set of these labels $W_\garbler$ and \broker has a wire
label $w_b$, we call the set of these labels $W_\broker$. Input consistency
check ensures that $w_b$ is either $w_0$ or $w_1$, otherwise the \broker would
be unable to evaluate the garbled circuit.

\begin{figure}[h]
	\centering
\begin{mdframed}[style=myframe]

\begin{itemize}[leftmargin=*,itemsep=4pt]

	\item[] For every input bit $b$ of circuit $C$, \garbler has both wire labels
		$w_0$ and $w_1$, we call the set of these labels $W_\garbler$ and \broker
		has a wire label $w_b$, we call the set of these labels $W_\broker$. Input
		consistency check ensures that $w_b$ is either $w_0$ or $w_1$.
	%	, otherwise the \broker would be unable to evaluate the garbled circuit.

	\item \broker and \garbler share an ephemeral seed $s''$ for wire label
		consistency check before every computation $C$.

	\item \broker generates a pseudorandom mask $r_b$, using seed $s''$, and
		computes $w_b \xor r_b$. We call the set of these masked input labels
		$R_\broker$. 

	\item \garbler generates the same pseudorandom mask $r_b$, using seed
		$s''$, and computes $w_0 \xor r_b$ and $w_1 \xor r_b$. We call the set of
		these masked labels $R_\garbler$.

	\item \broker and \garbler runs private-set-intersection (PSI) protocol with
		$R_\broker$ and $R_\garbler$ as inputs and only \broker receives the
		intersection.
		
	\item	If cardinality of $R_\broker \cap R_\garbler$ is $|R_\broker| =
		|R_\garbler / 2|$, then all wire labels are consistent. 

		\item If cardinality of $R_\broker \cap R_\garbler$ is less than
			$|R_\broker| = |R_\garbler / 2|$, then $|R_\broker| - |R_\broker \cap
			R_\garbler|$ labels are inconsistent. In this case:

			\begin{itemize}[leftmargin=*,itemsep=4pt,topsep=4pt]

					\item \broker unmasks all masked labels in $R_\broker \cap
						R_\garbler$ to learn $W_\broker \cap W_\garbler$.

					\item \broker sends all inconsistent labels $W_\broker \setminus
						(W_\broker \cap W_\garbler)$ to \garbler. 
						
					\item \garbler hard-wires in the garbled circuit $GC$ nullifying
						values for inputs of all publishers with at least $1$ inconsistent
						label.
						
			\end{itemize}
			
\end{itemize}

\end{mdframed}
\caption{Efficient Wire Label Consistency Checking Protocol}
\label{fig:label_consistency}
\end{figure}



\paragraph{Garbled Circuit XOR Compatibility.}


* What if one publisher doesn't send wire labels. After timeout the broker can
inform the \garbler and it will use zero for such values in the circuit.

\improvement{We can hide $C$ using universal circuits.}
\unsure{Who forms the topics? Publishers, clients, or broker?}


\improvement{Using cut and choose for malicious security}

\unsure{What is our current model? Are publishers, subscribers malicious?}

\improvement{A publisher can always input wrong data, but there is nothing that
can be done for this? May be it could be detected if it is creating a real
problem.}

\improvement{A subscribing entity can setup it's own \garbler. First we explain
with \garbler as a separate tntity and then later explain how it can be
eliminated by transfering it's functionality to subscribers. Similary, we can
transfer some functionality to publishers. Which one of them is better?}
