\begin{figure}
	\centering
\begin{mdframed}[style=myframe]

\begin{itemize}[leftmargin=*,itemsep=4pt]

	\item[] For every input bit $b$ of circuit $C$, \garbler has both wire labels
		$w_0$ and $w_1$, we call the set of these labels $W_\garbler$ and \broker
		has a wire label $w_b$, we call the set of these labels $W_\broker$. Input
		consistency check ensures that $w_b$ is either $w_0$ or $w_1$.
	%	, otherwise the \broker would be unable to evaluate the garbled circuit.

	\item \broker and \garbler share an ephemeral seed $s''$ for wire label
		consistency check before every computation $C$.

	\item \broker generates a pseudorandom mask $r_b$, using seed $s''$, and
		computes $w_b \xor r_b$. We call the set of these masked input labels
		$R_\broker$. 

	\item \garbler generates the same pseudorandom mask $r_b$, using seed
		$s''$, and computes $w_0 \xor r_b$ and $w_1 \xor r_b$. We call the set of
		these masked labels $R_\garbler$.

	\item \broker and \garbler runs private-set-intersection (PSI) protocol with
		$R_\broker$ and $R_\garbler$ as inputs and only \broker receives the
		intersection.
		
	\item	If cardinality of $R_\broker \cap R_\garbler$ is $|R_\broker| =
		|R_\garbler / 2|$, then all wire labels are consistent. 

		\item If cardinality of $R_\broker \cap R_\garbler$ is less than
			$|R_\broker| = |R_\garbler / 2|$, then $|R_\broker| - |R_\broker \cap
			R_\garbler|$ labels are inconsistent. In this case:

			\begin{itemize}[leftmargin=*,itemsep=4pt,topsep=4pt]

					\item \broker unmasks all masked labels in $R_\broker \cap
						R_\garbler$ to learn $W_\broker \cap W_\garbler$.

					\item \broker sends all inconsistent labels $W_\broker \setminus
						(W_\broker \cap W_\garbler)$ to \garbler. 
						
					\item \garbler hard-wires in the garbled circuit $GC$ nullifying
						values for inputs of all publishers with at least $1$ inconsistent
						label.
						
			\end{itemize}
			
\end{itemize}

\end{mdframed}
\caption{Efficient Wire Label Consistency Checking Protocol}
\label{fig:label_consistency}
\end{figure}
