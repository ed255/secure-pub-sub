\section{System}
\label{sec:system}

Our protocol implementation works on top of the IoT \MQTT{} protocol, which
allows exchanging messages in a publisher-subscriber model arbitrated by a
broker.  In order to set up the ratchet keys, both Subscribers and Publishers
need a private communication channel with the Third Party.  In order to embed
this communication into the MQTT protocol each client (either Publisher or
Subscriber) will have a device-specific topic that will allow two-way
authenticated communication between each client and the Broker.  The Broker
will forward the messages from this channel to the Third Party so that clients
can establish a ratchet key with the Third Party.  This design choice helps us
maintain the \MQTT{} semantics on the clients, since they only exchange
messages with a single Broker at the \MQTT{} level.

Although \MQTT{} can be used on top of TLS to provide secrecy, authenticity and
forward-secrecy; when clients obtain the ratchet keys from the Third Party
the communication is forwarded by the Broker at the application layer, so we
need to protect it to maintain secrecy, authenticity and forward-secrecy
against the Broker.  For this reason we choose to add authentication in the
\MQTT{} messages by means of public key signature and we use a key exchange
protocol to share secrets between Clients and the Third Party.  In the case of
Publishers, this authenticated key exchange is used to derive the Publisher'
ratchet key.  In the case of Subscribers, for every function subscription, a
key exchange is performed to derive a key to encrypt the function ratchet key
sent from the Third Party to the Subscriber.

This design achieves the desired cryptographic properties of TLS for selected
parts of the messages underlying the \MQTT{} communication, allowing the Broker
to be aware of the operation without being able to reveal or tamper any ratchet
key.

During the Publisher setup phase, not only its ratchet key will be established
but also the Broker will register a special publishing topic that the Publisher
will use to submit its encrypted values.  Every time the Broker evaluates a
garbled circuit using Publishers encrypted values as inputs, it will publish
the masked result to the relevant function topic so that the interested
Subscribers can receive it after having subscribed to that function topic.

The \MQTT{} messages for our protocol are encoded in JSON, with all the binary
data encoded in base64.  A possible improvement in performance and bandwidth
usage would consist on replacing this encoding for a binary one such as
Protocol Buffers.

\subsection{Extending \libgarble.}

\libgarble{} is a garbling library written in C based on \emph{JustGarble} that
is in current development.  It extends \emph{JustGarble} by adding two
different size optimizations in the garbled circuit: Half-gates (which combines
the free-XOR optimization with AND gates that only require two ciphertexts) and
Privacy-free garbling (which achieves the free-XOR optimization with AND gates
that only require one ciphertext at the cost of losing privacy and only
guaranteeing authenticity).  Apart from restructuring the code to make it more
concise, it also offers a better API than \emph{JustGarble} to build circuits.

Nevertheless, \libgarble{} being in current development lacks some
functionality which we had to add in order to build circuits and garble them
according to our needs.  On the garbling side, we implemented the NOT gate
(expressed as the XOR of the input with 1 to take advantage of the free-XOR
optimization) and the OR gate.  Our implementation doesn't use the half-gates
optimization due to the fact for this mode of operation, \libgarble{} in its
current design only reserves two ciphertext per gate, not allowing the
implementation of OR gates in a straight-forward manner.

We added some arithmetic blocks to be used when building circuits in order to
allow a wide variety of functions to be described: \textbf{signed fixed point
multiplication}, \textbf{signed fixed point division} and \textbf{signed
min/max}.

The motivation to operate with fixed point numbers was to be able apply
arbitrary functions based on arithmetic operations without the constraint of
having just integer values.  Even though it would be easy to scale input values
to be integers maintaining the desired precision, it's common in several
algorithms used in our applications to have intermediate values with decimals
that are significant.

A similar reasoning goes for supporting signed numbers.  We use the common
two's complement to represent such numbers, which work straightforward for
addition and integer multiplication but require special changes for fixed point
multiplication, division and min/max.

Whereas expanding \textbf{multiplication} and \textbf{min/max} to support
signed fixed point numbers was not hard (for multiplication we need to fix the
scaling of the result and for min/max we need to figure out the signedness of
the inputs), implementing \textbf{division} becomes a challenge.  Most of the
fast division algorithms are not fit for combinational circuits, which is a
constraint for the garbled circuits, because they assume circuit loops which we
will need to unroll, among other optimizations.  For this reason we implemented
the simplest non-optimized division algorithm: division by repeated subtraction
with a fixed number of iterations (assuming the worst case).

\subsection{High level language to build circuits.}

While \libgarble{} exposes the functions needed to build circuits corresponding
to the functions we are interested in, the process becomes cumbersome and error
prone.  To facilitate building circuits from functions we have implemented an
interpreter for a functional programming language inspired in Scheme, a dialect
of Lisp.  This choice was motivated both by the simplicity of parsing a
Lisp-like language as well as the similarity between the descriptive nature of
functional languages with combinational circuits, which helps reasoning about
how the circuit is built.  Moreover, the availability of high-order functions
in the language (functions that take functions as arguments and/or return
functions as a result) allows us to express circuits that combine building
blocks very concisely.

The interpreter is written in golang in just 470 lines of code.

Other solutions to express garbled circuits in higher level languages than the
underlying gates themselves have been proposed: namely \OblivC{}.  \OblivC{}
allows the developer to embed the secure computation part of a protocol in C
code with a GCC wrapper that takes care of the compilation, the circuit
building and the integration in a distributed program.  We don't require this
embedding in our protocol because the secure function is directly described
independently.

% TODO: Add Obliv-C reference

\lstinputlisting[style=Scheme, frame=single,
caption={Definition of the \emph{fold} high order function with an example of
its usage with \emph{min2} to define a new function.}]
{listings/prelude.lsp}

Where the only \libgarble{} building block is \texttt{min2}, which takes two
signed integers and returns the smaller one.

\texttt{fold} is a high order function that recursively traverses a list
\texttt{l} combining its elements with a function \texttt{f} to build up a
return value.

\lstinputlisting[style=Scheme, frame=single,
caption={Example of a function over Publishers' values.}]
{listings/min-example.lsp}

From this function definition the interpreter is capable of constructing the
circuit that evaluates the function as well as storing the list of publisher
values required to evaluate it.

Four our implementation we decided to use the same representation for all
numerical values: two's complement fixed point 32 bit numbers, with 26 bits for
the integer part and 8 bits for the decimal part.

\subsection{Broker and Third Party.}

The Broker and the Third Party have been written in the Go programming
language.  The choice of go was motivated on one hand due to the provided
built-in concurrency facilities such as \emph{goroutines} (light-weight
threads) and \emph{channels} (a primitive to send messages between
\emph{goroutines}), which are very appropriate for the networking server nature
of both entities.  On the other hand, this language offers the good performance
of a compiled language with the benefits of memory safety (which is very
welcomed when dealing with dynamic data structures).

The communication between the Broker and Third Party (to request the garbling
of a circuit for Publishers inputs at some particular rounds) happens over RPC
synchronously.

Implementation wise, the Broker requires a regular \MQTT{} Broker server
{reference: https://github.com/jeffallen/mqtt} with the minor modification that
any received message whose topic starts with a name space reserved for our
protocol, will be handled according to the protocol description instead of
being forwarded to the subscribers of that topic.

The Broker and Third Party are required to maintain replicated data structures
to store information of every Publisher, Subscriber and function.  Our design
makes this requirement easy because all setup messages between clients and
Third Party are forwarded by the Broker, allowing the later one to register the
non-secret details of the setup.

The Third Party is in charge of maintaining Publishers ratchet keys in sync
with Publishers, and Function ratchet keys in sync with Subscribers interested
in that function.  The Third Party also garbles circuits on demand for the
Broker.

The Broker is in charge of storing Publishers encrypted values with their
corresponding publisher ratchet key round, and of requesting the garbling of
circuits to the Third Party (for those particular Publishers values rounds) to
evaluate the garbled circuit and send the result to the subscribed Subscribers.

\subsection{Publisher and Subscriber.}

The Publisher has been written in C in order to minimize its memory footprint
as well as its resource consumption, considering that the code may be running
in a resource-constraint IoT device.  For the \MQTT{} part of the Publisher we
have used the Eclipse Paho client library {reference:
https://www.eclipse.org/paho/clients/c/}.


On the other hand, the Subscriber has been written in golang, assuming that it
will be running on a more powerful device.  The \MQTT{} library used by the
Subscriber is the same we used in the Broker, which also provides client
functionality {reference: https://github.com/jeffallen/mqtt}.

Publishers and Subscribers follow a state machine to initialize themselves
(i.e.\ for Publishers to obtain a Publisher ratchet key, for Subscribers to
obtain the Function ratchet key of all desired functions).  The initialization
to establish the ratchet keys happens through a message exchange with the
Broker and Third Party via \MQTT{} messages published and received at a unique
per-device specific topic.  The device specific topic is formed using the
Client's (Subscriber or Publisher) public key.

\subsection{Identity Gates for FreeXOR Compatibility.}

The free XOR optimization in garbled circuits require that for every wire, the
XOR of the label for bit 0 and bit 1 be a unique secret constant \emph{delta}
value.  Since input values are generated by independent Publishers using their
own seed, it's not easy to achieve this pattern without adding complexity to
the system.  For this reason, the Third Party will generate one \emph{identity}
gate per input wire that will allow transforming the inputs sent by Publishers
to garbled circuit inputs that follow the \emph{delta} pattern.

The implementation of these identity gates has been written in Go and is
integrated in the Broker and Third Party.  Because it's not written in C, like
the rest of the circuit garbling and garbled circuit evaluation, we consider
this step not to be optimized for speed.  Nevertheless, its computation
overhead should be small and linear in the number of the circuit function
inputs.

% OTHER STUFF

\subsection{Cryptographic choices}

For all the cryptographic operations we have used the \emph{libsodium} C
library {reference: https://download.libsodium.org/doc/} which is a fork of
NaCl {reference: http://nacl.cr.yp.to/}.  A particular feature of this library
is that it offers basic cryptographic primitives with secure design choices
for the underlying construction, algorithms and parameters.  This provides a
high level of usability and removes the possibility of some error-prone
choices.

In particular, the library makes the following choices for the primitives we use.

\begin{itemize}
  \item \textbf{Key Derivation Function}: \emph{BLAKE2B} is used for the
    underlying hash function.
  \item \textbf{Public key signatures}: \emph{Ed25519} elliptic curves are used.
  \item \textbf{Key Exchange}: Elliptic curve Diffie-Hellman with the
    \emph{Curve25519} curve (that is, \emph{X25519}) is used to obtain a shared
    secret, from which cryptographically secure keys are derived by applying
    the \emph{BLAKE2B-512} hash function.
  \item \textbf{Encryption}: The \emph{ChaCha20} stream cipher is used for
    encryption with \emph{Poly1305} MAC to provide authenticity.
  \item \textbf{CPRNG}: Uses the \emph{ChaCha20} stream cipher.
\end{itemize}

Even though \emph{libsodium} allows us to use \emph{AES} for encryption, we
choose to stick with \emph{ChaCha20} because it is faster in software-only
implementations (which is the only option available in many IoT devices) and
because it is not sensitive to cache-collision timing attacks by design.

The construction \emph{ChaCha20-Poly1305} has also been proposed as a standard
choice of ciphersuit in the upcoming TLS version 1.3, which gives us further
confidence in the decision to use it in our protocol.
